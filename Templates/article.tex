\documentclass[10pt,letterpaper]{article}
\usepackage{amsmath, amssymb, amsbsy}
\usepackage{cite, graphicx}
\usepackage{geometry}

% see http://en.wikibooks.org/wiki/LaTeX/Colors
% e.g. \color{red}{some text} to make "some text" red
\usepackage[usenames,dvipsnames]{xcolor} 

% used for sidecaptions; replace figure environment with SCFigure environment
\usepackage{sidecap} 

% if you want text to wrap around figures use wrapfigure environment
\usepackage{wrapfig}

% is you want to make figures with multiple pictures
\usepackage[font={small},margin=10pt]{caption}
\usepackage{subcaption}

% if you want to import code into latex (e.g. in an appendix)
% http://en.wikibooks.org/wiki/LaTeX/Packages/Listings
% e.g. \lstinputlisting[language=Python]{source.py}
\usepackage{listings}
\lstset{fancyvrb=true}
\lstset{
    basicstyle=\small\tt,
    keywordstyle=\color{blue},
    identifierstyle=,
    commentstyle=\color{orange},
    stringstyle=\color{red},
    showstringspaces=false,
    tabsize=2,
    numbers=left,
    captionpos=b,
    numerstyle=\tiny
}

\geometry{letterpaper,nohead,margin=1in}
\parindent1em
\parskip0pc
\linespread{1.0}
\pagestyle{plain}

\newcommand{\com}[1]{\hspace{2em}\textrm{#1}} % comments
\newcommand{\sinc}[0]{\textrm{sinc}}
\newcommand{\vv}[1]{\boldsymbold{#1}} % vector

\title{}
\date{}
\author{John David Giese}

\begin{document}
%\maketitle
\section*{Example Section}
\noident

%% NORMAL FIGURE
%\begin{figure}[htb]
    %\centerline{\includegraphics[width=0.9\textwidth]{example.pdf}}
    %\caption{Caption text goes here.}
    %\label{fig:exam} % reference with \ref{fig:exam} or \pageref{fig:exam}
%\end{figure}

%% FIGURE WITH SIDECAPTION
%\begin{SCfigure}[htb]
    %\centerline{\includegraphics[width=0.9\textwidth]{example.pdf}}
    %\caption{Caption text goes here.}
    %\label{fig:exam} % reference with \ref{fig:exam} or \pageref{fig:exam}
%\end{SCfigure}

%% WRAP FIGURE -- text goes around it
%\begin{wrapfigure}{r}{0.5\textwidth}
    %\vspace{-20pt}
    %\begin{center}
        %\includegraphics[width=0.48\textwidth]{example.pdf}
    %\end{center}
    %\vspace{-20pt}
    %\caption{example caption}
    %\vspace{-10pt}
    %\label{marker} % reference with \ref{marker} or \pageref{marker}
%\end{wrapfigure}

%% FIGURE WITH THREE SUBFIGURES
%\begin{figure}
    %\centering
    %\begin{subfigure}[b]{0.3\textwidth}
        %\centering
        %\includegraphics[width=\textwidth]{example1.pdf}
        %\caption{This is the first figure.}
        %\label{fig:sub1}
    %\end{subfigure}%
    %~ %add desired spacing between images, e. g. ~, \quad, \qquad etc. 
      %%(or a blank line to force the subfigure onto a new line)
    %\begin{subfigure}[b]{0.3\textwidth}
        %\centering
        %\includegraphics[width=\textwidth]{example2.pdf}
        %\caption{This is the second figure.}
        %\label{fig:sub2}
    %\end{subfigure}
    %~ %add desired spacing between images, e. g. ~, \quad, \qquad etc. 
      %%(or a blank line to force the subfigure onto a new line)
    %\begin{subfigure}[b]{0.3\textwidth}
        %\centering
        %\includegraphics[width=\textwidth]{example3.pdf}
        %\caption{This is the third figure.}
        %\label{fig:sub3}
    %\end{subfigure}
    %\caption{The caption for the entire figure goes here.}\label{fig:all3}
%\end{figure}

%% INKSCAPE PDF+Latex
%\begin{figure}[htb]
    %\centering
    %\def\svgwidth{6in}
    %\input{example.pdf_tex}
    %\caption{Caption text goes here.}
    %\label{marker} % reference with \ref{marker} or \pageref{marker}
%\end{figure}

\end{document}

